\documentclass{beamer}

\usepackage[russian]{babel}
\usepackage[utf8]{inputenc}
\usepackage{cmap}
\usepackage{graphicx}
\usepackage{xspace}
\usepackage{psfrag}

\newcommand{\MARK}[1]{{\bf {\it #1}}}
\newcommand{\CODE}[1]{{\ttfamily #1}}

\setbeamertemplate{footline}[frame number]
\usecolortheme{seahorse}
\beamertemplateshadingbackground{white}{blue!3}

\begin{document}

\begin{frame}
\begin{center}
Кирюшкина Валентина\\
\vspace{1cm}
{\Large Обзор алгоритмов извлечения информации из архива англоязычной Википедии.}
\end{center}
\end{frame}
\begin{frame}
\frametitle{Data mining и Text Mining}
Data mining – процесс нахождения полезных закономерностей (useful patterns)  в большом наборе данных.

\vspace{1cm}

Text Mining = Text Data Mining.
\end{frame}
\begin{frame}
\frametitle{Подходы к Data Mining}
\begin{enumerate}
\item{Классификация}
\item{Кластеризация}
\item{Извлечение правил ассоциаций}
\item{Выборка атрибутов}

\end{enumerate}
\end{frame}
\begin{frame}
\frametitle{Кластеризация для документов}
D $\to$ P, P $\in \mathbb{P}$
\vspace{1cm}
$\mathbb{P}$ - множество кластеров.\\

Документы разделяются на подмножества таким образом, что:

\begin{itemize}
\item{внутри одного подмножества документы \underline{однородны}}
\item{документы из разных подмножеств \underline{разнородны} }
\end{itemize}
\vspace{1cm}
Dist($d_{1}$,$d_{2}$) - мера расстояния между документами.
\end{frame}
\begin{frame}
\frametitle{Классификация документов}
D $\to$ L, L $\in \mathbb{P}$\\
\vspace{1cm}
$\mathbb{P}$ - множество категорий документов.\\
\vspace{1cm}
В отличие от кластеризации множество категорий задано изначально.\\
\vspace{1cm}
Дан обучающий набор образцов документов, для которых известно, какому классу они принадлежат.
\end{frame}
\begin{frame}
\frametitle{Объекты и атрибуты}
Нередко данные  об объекте хранятся в виде фиксированного списка атрибутов.\\
\vspace{1cm}
Опишем базу данных, которую мы будем использовать на следующих слайдах в качестве примера.\\
\vspace{1cm}
Возьмем в качестве рассматриваемых объектов разных животных.\\
Список атрибутов пусть будет следующим:\\
вид(строка), средний вес(число), кормление молоком(булевый), наличие шерсти(булевый), длина названия вида(число).\\
\end{frame}
\begin{frame}
\frametitle{Извлечение правил ассоциаций}
Основной задачей является поиск интересных и полезных закономерностей в базе данных.\\
\vspace{1cm}
{\bf Пример.}
Выберем два атрибута: средний вес и наличие шерсти.\\
Можем получить следующее логическое выражение:\\
если средний вес = 50 кг и шерсть = 1 $\to$ вид = волк или вид = шипанзе.\\
И так далее.\\
\end{frame}
\begin{frame}
\frametitle{Выборка атрибутов}
Задача сортировки атрибутов объектов по их информационной содержательности.\\
\vspace{1cm}
{\bf Пример.}
В нашей базе данных в результате действия алгоритмов выборки атрибутов получим отсортированный список: вид, средний вес, кормление молоком, наличие шерсти.\\
Очевидно, что атрибут «длина названия вида» не несет никакой информационной ценности.\\
\end{frame}
\begin{frame}
\frametitle{Weka}
\underline{Weka} - это cвободное программное обеспечение для анализа данных, написанное на Java в университете Уайкато (Новая Зеландия).\\

\vspace{1cm}

Weka позволяет выполнять такие важные для нас задачи анализа данных, как извлечение правил ассоциаций и выборка атрибутов.\\
В качестве источника данных могут использоваться реляционные базы данных или структурированные данные в формате CSV.\\
\end{frame}
\begin{frame}
\frametitle{Цель работы}
Создание базы данных на основе текстов англоязычной Википедии для запуска алгоритмов Weka.
\end{frame}
\begin{frame}
\frametitle{Гипотеза}
\begin{enumerate}
\item{Объектами БД будем считать страницы Википедии.}
   \item{Атрибуты можно получить на основе второстепенных членов предложения, в которых сам объект является подлежащим и употребляется в качестве ссылки.}
\end{enumerate}
\end{frame}
\begin{frame}
\frametitle{Пример}
The Doors' fourth album, \underline{The Soft Parade}, \underline{released} in July 1969, container pop-oriented sections. {\it (The Doors Wikipedia page)}\\

\vspace{1cm}

\underline{A sauceful of secrets} was \underline{released} in June 1968 {\it(Pink Floyd Wikipedia page)}

\end{frame}
\begin{frame}
\frametitle{Сложности}
\begin{enumerate}
\item{Необходимо выбрать фиксированный список атрибутов, наиболее точно отражающий суть объекта.}
\item{Необходимо удостовериться, что алгоритмы Weka справятся с БД подобного размера (проанализировать их трудоемкость).}
\end{enumerate}
\end{frame}

\end{document}




