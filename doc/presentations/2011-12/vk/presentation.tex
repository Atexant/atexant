\documentclass{beamer}

\usepackage[russian]{babel}
\usepackage[utf8]{inputenc}
\usepackage{cmap}
\usepackage{graphicx}
\usepackage{xspace}
\usepackage{psfrag}

\newcommand{\MARK}[1]{{\bf {\it #1}}}
\newcommand{\CODE}[1]{{\ttfamily #1}}

\setbeamertemplate{footline}[frame number]
\usecolortheme{seahorse}
\beamertemplateshadingbackground{white}{blue!3}

\begin{document}
\sloppy

\begin{frame}
\begin{center}
Кирюшкина Валентина\\
\vspace{1cm}
{\Large Исследование алгоритмов построения базы данных для методов Data mining\\
на~основе материалов свободной энциклопедии ``Википедия''}
\end{center}
\end{frame}

\begin{frame}
\frametitle{Data mining и Text mining}
\MARK{Data mining} --- процесс выделения полезных закономерностей ({\it useful patterns})  
в~больших массивах данных.

\vspace{1cm}

\MARK{Text mining} --- процесс выделения полезных закономерностей ({\it useful patterns})  
в~больших массивах текста на~естественном языке.
\MARK{Text mining} = Text Data Mining.
\end{frame}

\begin{frame}
\frametitle{Основные направления Data mining}

Основные направления Data mining следующие:

\begin{enumerate}
\item{классификация;}
\item{кластеризация;}
\item{извлечение правил ассоциаций;}
\item{выборка атрибутов.}
\end{enumerate}

Приведённые направления носят в~основном эвристическую природу.
\end{frame}

\begin{frame}
\frametitle{Кластеризация документов}
$$D \to P, P \in \mathbb{P}, D \in \mathbb{D}$$

\vspace{1cm}

где $\mathbb{D}$ --- исходное множество документов, $\mathbb{P}$ --- множество выделенных кластеров.\\
Документы разделяются на~кластеры таким образом, что:

\begin{itemize}
\item{внутри одного кластера документы \underline{однородны};}
\item{документы из~разных кластеров \underline{разнородны}.}
\end{itemize}
\vspace{1cm}
%%Dist($d_{1}$,$d_{2}$) - мера расстояния между документами.
\end{frame}

\begin{frame}
\frametitle{Классификация документов}
$$D \to L, L \in \mathbb{L}, D \in \mathbb{D}$$

\vspace{1cm}

где $\mathbb{D}$ --- исходное множество документов, 
$\mathbb{L}$ --- множество классов документов.
В~отличие от~кластеризации, множество классов задано изначально.

\vspace{5mm}

Предполагается наличие обучающего набора образцов документов, 
для~которого доступна информация о~принадлежности к~классам.
\end{frame}

\begin{frame}
\frametitle{Объекты и атрибуты}

Для~развитых алгоритмов данные  об~объекте хранятся в виде фиксированного списка атрибутов.
Множество объектов составляет базу данных (БД).
Приведём пример подобной базы данных:

\begin{itemize}
\item {множество объектов --- множество видов животных;}
\vspace{1cm}
\item {вариант множества атрибутов:
вид~({\it str}),
средний~вес~({\it float}),
кормление~молоком~({\it bool}),
 наличие~шерсти~({\it bool}),
длина названия вида~({\it int}).
}
\end{itemize}
\end{frame}

\begin{frame}
\frametitle{Извлечение правил ассоциаций}

Извлечение правил ассоциаций заключается в~поиске интересных и полезных закономерностей в~БД.

\vspace{1cm}

\MARK{Пример:}
выберем два~атрибута: средний вес и наличие шерсти,
можем получить следующее логическое выражение:
если средний вес равен 50~кг, и есть шерсть, то вид --- волк или шимпанзе, и~т.~д.
\end{frame}

\begin{frame}
\frametitle{Выборка атрибутов}
Выборка атрибутов --- задача сортировки атрибутов объектов по~их~информационной содержательности.

\vspace{1cm}

\MARK{Пример:}
в~БД после работы алгоритма выборки атрибутов должны получить список:
вид,
средний вес,
кормление молоком,
наличие шерсти.
Здесь нет атрибута ``длина названия вида'',
т.к. он, очевидно, не~несет никакой информационной ценности.
\end{frame}

\begin{frame}
\frametitle{Проект Weka}
\MARK{Weka} --- это свободно распространяемый пакет для~анализа данных, 
написанный на~языке {\it Java} в~университете Уайкато (Новая Зеландия).

\vspace{1cm}

Weka позволяет выполнять такие важные для~нас задачи анализа данных, 
как извлечение правил ассоциаций и выборка атрибутов.
В~качестве источника данных могут использоваться реляционные БД или структурированные данные в~формате {\it CSV}.
\end{frame}

\begin{frame}
\frametitle{Цель работы}
Цель работы --- исследование алгоритмов создания БД на~основе текстов англоязычной Википедии в~формате,
пригодном для~использования в~качестве исходных данных для~алгоритмов Weka.

\vspace{5mm}

Википедия выступает как~сверхбольшой массив текста с~ссылками между статьями.
\end{frame}

\begin{frame}
\frametitle{Предлагаемый подход}
\begin{enumerate}
\item{
Множество объектов БД --- множество статей Википедии.
}

\item {
Атрибуты предлагается получать на~основе анализа второстепенных членов предложений, 
в~которых сам объект является подлежащим и употребляется в~качестве ссылки.
}
\end{enumerate}
\end{frame}

\begin{frame}
\frametitle{Пример}
The~Doors' fourth album, \underline{The~Soft Parade}, 
{\bf released} in~July~1969, 
container pop-oriented sections. 
({\it The~Doors Wikipedia page})

\vspace{1cm}

\underline{A~sauceful of~secrets} was {\bf released} in~June~1968. 
({\it(Pink Floyd Wikipedia page)})
\end{frame}

\begin{frame}
\frametitle{Ожидаемые сложности}

Возможные ожидаемые сложности работы:

\begin{enumerate}

\item{
Сверхбольшие объёмы данных.
}

\item{
Низкий уровень статистической значимости выделенных атрибутов.
}

\end{enumerate}
\end{frame}

\begin{frame}
{\Large Спасибо за внимание!}
\end{frame}

\end{document}
