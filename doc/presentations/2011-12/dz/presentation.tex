\documentclass{beamer}

\usepackage[russian]{babel}
\usepackage[utf8]{inputenc}
\usepackage{cmap}
\usepackage{graphicx}
\usepackage{xspace}
\usepackage{psfrag}

\newcommand{\MARK}[1]{{\bf {\it #1}}}
\newcommand{\CODE}[1]{{\ttfamily #1}}

\setbeamertemplate{footline}[frame number]
\usecolortheme{seahorse}
\beamertemplateshadingbackground{white}{blue!3}

\begin{document}

\begin{frame}
\begin{center}
Жарков Денис\\
\vspace{1cm}
{\Large Исследования алгоритмов кластеризации статей свободной энциклопедии ``Википедия'' 
на~основе ссылок и второстепенных членов предложений}\\
\end{center}
\end{frame}

\begin{frame}
\frametitle{Основная задача}
Множество статей Википедии рассматривается как множество понятий.
В~начальном состоянии нет~никакой информации о~семантике понятий. 

\vspace{1cm}

В~ходе работы проводится выделение кластеров понятий, каждому из~которых далее указывается некоторая семантика.
В~научной литературе подобный процесс обычно называется ``кластеризацией''.
\end{frame}

\begin{frame}
\frametitle{Особенности работы}
\begin{enumerate}
\item {
В~качестве исходных данных Википедия выступает как~большой массив структурированного текста со~связями между статьями.
}

\item {
Для~вспомогательных задач по~обработке естественного языка (получение нормальной формы слов и~пр.) 
используется библиотека Стэнфордского университета \MARK{CoreNLP}.
}

\item{
Процесс кластеризации носит эвристический характер.
}
\end{enumerate}
\end{frame}

\begin{frame}
\frametitle{Полезность работы}

\begin{itemize}
\item {
Определяется принадлежность различных имён собственных некоторому интересующему кластеру,
которая затем может использоваться в~эвристической оценке содержания текста (например, при спам-фильтрации).
}

\item {
Множеству кластеров понятий назначается некоторая семантика, которая затем может использоваться в~качестве основы более сложных алгоритмов.
Предполагается, что наличие текста само по~себе не~даёт семантической ценности.
}
\end{itemize}

\end{frame}

\begin{frame}
\frametitle{Пример}
\begin{itemize}

\item {
В~двадцатом столетии в~музее {\bf выставлялась} \underline{Мона Лиза} Леонардо да~Винчи. {\it (из~статьи о~Пушкинском музее)}
}

\item {
\underline{Герника} --- громадное полотно, которое было {\bf выставлено} в~республиканском павильоне Испании. {\it (из~статьи о~Пабло Пикассо)}
}

\end{itemize}
\end{frame}

\begin{frame}
\frametitle{Подзадачи}
\begin{enumerate}

\item {
Выделение предложений со~ссылками из~дампа;
}
\item {
морфологический анализ отдельного предложения;
}
\item {
сохранение наиболее часто используемых слов в качестве второстепенных членов к~каждой статье;
}
\item {
фильтрация общих/малозначимых слов-среди второстепенных членов;
}
\item {
кластеризация массива статей;
}
\item {
проверка адекватности полученных кластеров.
}

\end{enumerate}
\end{frame}

\begin{frame}
\frametitle{Сложности}
\begin{enumerate}

\item {
Википедия содержит 11 миллионов страниц (Около 3 миллионов статей).
}
\item {
Необходимо придумать адекватное векторное представление документов и меру расстояния.
}
\item {
Нужно найти ``эталон'' ккластеризации, с~которым можно будет сравнивать результаты.
}

\end{enumerate}
\end{frame}

\end{document}
