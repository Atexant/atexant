\documentclass{beamer}

\usepackage[russian]{babel}
\usepackage[utf8]{inputenc}
\usepackage{cmap}
\usepackage{graphicx}
\usepackage{xspace}
\usepackage{psfrag}

\newcommand{\MARK}[1]{{\bf {\it #1}}}
\newcommand{\CODE}[1]{{\ttfamily #1}}

\setbeamertemplate{footline}[frame number]
\usecolortheme{seahorse}
\beamertemplateshadingbackground{white}{blue!3}

\begin{document}
\sloppy

\begin{frame}
\begin{center}
Жарков Денис\\
\vspace{1cm}
{\Large Исследования алгоритмов кластеризации статей свободной энциклопедии ``Википедия'' 
на~основе ссылок и второстепенных членов предложений}
\end{center}
\end{frame}

\begin{frame}
\frametitle{Основная задача}
Множество статей Википедии рассматривается как множество понятий.
В~начальном состоянии отсутствует информация о~семантике понятий. 

\vspace{1cm}

В~ходе работы проводится выделение кластеров понятий, каждому из~которых далее указывается некоторая семантика.
В~научной литературе по~Data mining подобный процесс обычно называется ``кластеризацией''.
\end{frame}

\begin{frame}
\frametitle{Особенности работы}
\begin{enumerate}
\item {
В~качестве исходных данных Википедия выступает как~большой массив структурированного текста со~связями между статьями.
}
\item {
Для~вспомогательных задач по~обработке естественного языка (получение нормальной формы слов и~пр.) 
используется библиотека Стэнфордского университета \MARK{CoreNLP}.
}
\item{
Процесс кластеризации носит эвристический характер.
}
\end{enumerate}
\end{frame}

\begin{frame}
\frametitle{Полезность работы}
Ценность работы заключается:
\begin{itemize}
\item {
в~определении информации о~принадлежности различных имён собственных некоторому интересующему кластеру,
которая затем может использоваться в~эвристической оценке содержания текста (например, при спам-фильтрации).
}

\item {
в~назначении множеству кластеров понятий некоторой семантики, 
которая затем может выступать в~качестве основы более сложных алгоритмов.
}
\end{itemize}
\end{frame}

\begin{frame}
\frametitle{Подход к~решению задачи}
Предлагается выполнять кластеризацию на~основе введённого понятия близости между объектами,
в~качестве которых выступают статьи Википедии.

\vspace{1cm} 

Необходимо иметь возможность определения соответствия между подлежащими в~предложениях
и множеством объектов (статей).
Понятие близости строиться на~основе анализа частоты употребления одинаковых второстепенных членов для~каждого подлежащего.
\end{frame}

\begin{frame}
\frametitle{Сопоставление подлежащих и объектов}

Подлежащее считается сопоставленным с~некоторым объектом (статьёй), если:

\begin{itemize}
\item {
является ссылкой,
и в~этом случае объектом считается статья, на~которую указывает ссылка;
}
\item {
в~нормальнойй форме совпадает с~названием статьи и ссылкой не~является,
в~этом случае объектом считается статья, в~которой находится подлежащее.
}
\end{itemize}
\end{frame}

\begin{frame}
\frametitle{Пример}
\begin{itemize}
\item {
В~двадцатом столетии в~музее {\bf выставлялась} \underline{Мона Лиза} Леонардо да~Винчи. 
({\it из~статьи о~Пушкинском музее})
}

\item {
\underline{Герника} --- громадное полотно, которое было {\bf выставлено} в~республиканском павильоне Испании. 
({\it из~статьи о~Пабло Пикассо})
}
\end{itemize}
\end{frame}

\begin{frame}
\frametitle{Вспомогательные операции}
\begin{enumerate}
\item {
Выделение предложений со~ссылками из~дампа Википедии в~формате {\it xml}.
}
\item {
Морфологический анализ отдельного предложения.
}
\item {
Сохранение наиболее часто используемых слов в качестве второстепенных членов к~каждой статье.
}
\item {
Фильтрация общих/малозначимых слов-среди второстепенных членов.
}
\item {
Кластеризация массива объектов (статей).
}
\item {
Проверка адекватности полученных кластеров.
}
\end{enumerate}
\end{frame}

\begin{frame}
\frametitle{Возможнные трудности}
\begin{enumerate}
\item {
Сверхбольшие объёмы данных --- Википедия содержит 11~миллионов страниц (Около 3~миллионов статей).
}
\item {
Сложности в~подборе правил для~удаления малозначимых слов.
}
\item {
Поиск некоторого уже существующего решения,
пригодного для~рассмотрения в~качестве эталона для~проверки состоятельности результатов 
(рассматривается возможность использования встроенных категорий Википедии). 
}
\end{enumerate}
\end{frame}

\begin{frame}
{\Large Спасибо за внимание!}
\end{frame}

\end{document}
